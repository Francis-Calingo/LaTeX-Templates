\documentclass[12pt]{amsart}
\usepackage{geometry}                % See geometry.pdf to learn the layout options. There are lots.
\geometry{letterpaper}                   % ... or a4paper or a5paper or ... 



\usepackage{tikz}			    % This package allows us to do a LOT of graphics stuff. Google Tikz
\newcommand{\red}[1]{{\color{red} #1}}

\setlength{\evensidemargin}{.75in}
\addtolength{\evensidemargin}{-1in}
\setlength{\oddsidemargin}{.5in}
\addtolength{\oddsidemargin}{-1in}
\setlength{\topmargin}{.75in}
\addtolength{\topmargin}{-1.25in}
\setlength{\textwidth}{7.5in}
\setlength{\textheight}{10in}
\linespread{1}


\pagestyle{empty}
\setlength{\oddsidemargin}{0.3in} 
\setlength{\textheight}{8in}
\setlength{\topmargin}{0.4in}
\setlength{\headsep}{0in}
\setlength{\parskip}{0pt}
\setlength{\parindent}{0pt}
\setlength{\textwidth}{5.9in}
\font\Bbb=msbm10                 %outline bold character 
\newcommand{\RR}{\hbox{\Bbb R}}
\newcommand{\CC}{\hbox{\Bbb C}}
\newcommand{\QQ}{\hbox{\Bbb Q}}
\newcommand{\NN}{\hbox{\Bbb N}}
\newcommand{\ZZ}{\hbox{\Bbb Z}}
\newcommand{\kk}{\hbox{\bf k}}
\newcounter{probnum}
\partopsep 0pt
\settowidth{\leftmargini}{1.}\addtolength{\leftmargini}{\labelsep}
\newenvironment{prob}{\begin{enumerate}\setcounter{enumi}{\value{probnum}}}%
 {\setcounter{probnum}{\value{enumi}}\end{enumerate}}
\settowidth{\leftmarginii}{(a)}\addtolength{\leftmarginii}{\labelsep}
\newenvironment{subprob}{\begin{enumerate}\topsep 0pt}{\end{enumerate}}
\renewcommand{\theenumiii}{\roman{enumiii}}
\renewcommand{\labelenumiii}{(\theenumiii)}
\settowidth{\leftmarginiii}{(viii)}\addtolength{\leftmarginiii}{\labelsep}
\newenvironment{ssubprob}{\begin{enumerate}}{\end{enumerate}}
\def\heading#1{\vskip\bigskipamount\hfill{\sc #1}\hfill}

\begin{document}

\begin{center}
{\large \sc 
PEER TUTOR INTERVIEW QUESTIONS} \\
\  \\ Francis Calingo\\
\\TOPIC: Inverse Matrices (MATH 1025) \\

 \end{center}
\vspace{.5in}

\begin{prob} 
\item[(1)]   Let $A$ be the following 3 x 3 matrix:
$$
\left[ 
\begin{array}{ccc}
3 & 1 & 1 \\
2 & 1 & 0 \\
1 & 1 & 1 \\
\end{array} \right]
$$
Also, let $C$ be some 3x3 matrix, \(C \neq A \). Solve for C, given the following equation:

$$
C*A^-^1=det(A)*I
$$

\red{Easiest way to determine determinant: det($A$)=(-1)^1^+^3[(2)(1)-(1)(1)]+(-1)^3^+^3[(3)(1)-(2)(1)]=1+1=2}

\red{Observe that $C$*$A^-^1$*$A$=$C$*$I$=$C$}.

\red{Therefore, $C=det(A)*I*A$=$det(A)*A$=}
\red{$$
\left[ 
\begin{array}{ccc}
6 & 2 & 2 \\
4 & 2 & 0 \\
2 & 2 & 2 \\
\end{array} \right]
$$}
 \end{prob}
 \bigskip
\begin{prob} 
\item[(2)]   Let $D$ be the matrix
 $$
\left[ 
\begin{array}{cc}
cos^2(\theta) & 1 \\
sin^2(\theta) & 1  \\
\end{array} \right]
$$
Also, let \(-\pi \leq \theta \leq \pi\).
Does $D$ have an inverse matrix? Explain. 

\red{A square matrix is invertible if and only if its determinat does not equal 0.}

\red{$det(D)$=$cos^2(\theta)-sin^2(\theta)=cos(2\theta)$}

\red{If $cos(2\theta)=0$, then the matrix is not invertible.}

\red{Recall that $cos(\theta)=0$ whenever \theta=\frac{\pi}{2}+n*\pi}, 

\red{n being any integer.}

\red{Therefore, it follows that $cos(2*\theta)=0$}

\red{whenever \theta=\frac{\pi}{4}+\frac{n*\pi}{2}}.

\red{Therefore, it follows that this matrix is not invertible if \theta=\frac{-3\pi}{4}, \frac{-\pi}{4}, \frac{\pi}{4}, or \frac{3\pi}{4}}.
 \end{prob}
\bigskip
\begin{prob} 
\item[(3)]   Let $G$ be the matrix 
$$
\left[ 
\begin{array}{ccc}
2 & 1 & 1 \\
4 & 1 & 1 \\
2 & 3 & 2 \\
\end{array} \right]
$$
\begin{subprob}
\item{}Find its adjugate matrix ($adj(G)$) and $det(G)$.

\red{adj(G)=

$$
\left[ 
\begin{array}{ccc}
(1)det(G_1_,_1) & (-1)det(G_1_,_2) & (1)det(G_1_,_3) \\
(-1)det(G_2_,_1) & (1)det(G_2_,_2) & (-1)det(G_2_,_3) \\
(1)det(G_3_,_1) & (-1)det(G_3_,_2) & (1)det(G_3_,_3) \\
\end{array} \right]^T
$$},

\red{where 
$$
G_1_,_1=\left[ 
\begin{array}{cc}
1 & 1 \\
3 & 2 \\
\end{array} \right], G_1_,_2=\left[ 
\begin{array}{cc}
4 & 1 \\
2 & 2 \\
\end{array} \right],  G_1_,_3=\left[ 
\begin{array}{cc}
4 & 1 \\
2 & 3 \\
\end{array} \right], G_2_,_1=\left[ 
\begin{array}{cc}
1 & 1 \\
3 & 2 \\
\end{array} \right],
$$}

\red{$$
G_2_,_2=\left[ 
\begin{array}{cc}
2 & 1 \\
2 & 2 \\
\end{array} \right], G_2_,_3=\left[ 
\begin{array}{cc}
2 & 1 \\
2 & 3 \\
\end{array} \right], G_3_,_1=\left[ 
\begin{array}{cc}
1 & 1 \\
1 & 1 \\
\end{array} \right], G_3_,_2=\left[ 
\begin{array}{cc}
2 & 1 \\
4 & 1 \\
\end{array} \right], G_3_,_3=\left[ 
\begin{array}{cc}
2 & 1 \\
4 & 1 \\
\end{array} \right].
$$}

\red{Therefore, adj(G)=
$$
\left[ 
\begin{array}{ccc}
(1)(-1) & (-1)(6) & (1)(10) \\
(-1)(-1) & (1)(2) & (-1)(4) \\
(1)(0) & (-1)(-2) & (1)(-2) \\
\end{array} \right]^T=\left[ 
\begin{array}{ccc}
-1 & -6 & 10 \\
1 & 2 & -4 \\
0 & 2 & -2 \\
\end{array} \right]^T=\left[ 
\begin{array}{ccc}
-1 & 1 & 0 \\
-6 & 2 & 2 \\
10 & -4 & -2 \\
\end{array} \right]
$$}

\red{Also, det(G)=(1)[(4)(3)-(2)(1)]+(-1)[(2)(3)-(2)(1)]+(2)[(2)(1)-(4)(1)]

=10-4-4=2}

\item{}Find $G^-^1$ using $adj(G)$ and $det(G)$.

\red{G^-^1=\frac{1}{det(G)}adj(G)=
\frac{1}{2}*$$
\left[ 
\begin{array}{ccc}
-1 & 1 & 0 \\
-6 & 2 & 2 \\
10 & -4 & -2 \\
\end{array} \right]
$$}

\end{subprob}
\end{prob}
\bigskip
\begin{prob} 
\item[(4)] Let $A$ and $B$ be 2x2 matrices. Prove that $(AB)^-^1=B^-^1A^-^1$.  

\red{This is one possible answer. Assume that the equation is true. Therefore, it follows that the inverse of AB is $B^-^1A^-^1$}.

\red{Observe that (AB)($B^-^1A^-^1$)

=A*B*$B^-^1$*$A^-^1$

=A*I*$A^-^1$

=A*$A^-^1$

=I,}

\red{as desired.}

\end{prob}
 \end{document}



