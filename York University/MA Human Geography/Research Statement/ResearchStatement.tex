\documentclass{article}

% Language setting
% Replace `english' with e.g. `spanish' to change the document language
\usepackage[english]{babel}

% Set page size and margins
% Replace `letterpaper' with`a4paper' for UK/EU standard size
\usepackage[letterpaper,top=2cm,bottom=2cm,left=3cm,right=3cm,marginparwidth=1.75cm]{geometry}

% Useful packages
\usepackage[colorlinks=true, allcolors=blue]{hyperref}
\setlength{\parindent}{0pt}
\usepackage{fancyhdr}
\title{\textbf{Research Statement-MA Geography, Fall 2025}}
\author{\textbf{Francis Emmanuel Calingo}}


\begin{document}

\maketitle

The Filipino diaspora’s accelerated growth in Toronto’s North York suburb, where they constitute the largest migrant group\footnote{\textit{Community Council Area Profiles (North York)}. City of Toronto, Copyright 2024, \url{https://www.toronto.ca/wp-content/uploads/2024/01/8ce5-CityPlanning-2021-Census-Profile-North-York-CCA.pdf} , p. 22
}, renders socioeconomic research on them, particularly education, needed. 

\vspace{10pt}

Second-generation Filipinos often achieve \textit{lower} educational attainment than their first-generation counterparts, impacting upward mobility.\footnote{Kelly, Philip F. \textit{Understanding Intergenerational Social Mobility: Filipino Youth in Canada}. IRPP, 2014.
} I hope to conduct research that asks what localized impacts socioeconomic issues pertinent to Filipino Canadians (e.g., remittances, racialization of certain jobs, deprofessionalization) have on this intergenerational gap? By implementing decolonized, Filipino-centred research frameworks that emphasize storytelling (\textit{kwentuhuan})\footnote{Gutierrez, Rose Ann E., et al. \textit{Co‐creating Knowledge with Undocumented Filipino Students: Kuwentuhan as a Research Method.} New Directions for Higher Education, vol. 2023, no. 203, Sept. 2023, pp. 77–92. DOI.org (Crossref), \url{https://doi.org/10.1002/he.20478}.} and key principles of \textit{Sikolohiyang Pilipino} (Filipino psychology) that equalizes subject and researcher dynamics (e.g., using the subject’s mother tongue)\footnote{Pe‐Pua, Rogelia, and Elizabeth A. Protacio‐Marcelino. \textit{Sikolohiyang Pilipino (Filipino Psychology): A Legacy of Virgilio G. Enriquez.} Asian Journal of Social Psychology, vol. 3, no. 1, Apr. 2000, p. 49-71. DOI.org (Crossref), \url{https://doi.org/10.1111/1467-839X.00054}.
}
, this project will collaborate with grassroots Filipino organizations that I have done fieldwork for-York’s Filipino association, Anakbayan Toronto, etc. It will also apply machine learning and sampling techniques on socioeconomic data from sources like Statistics Canada, IRCC, and interviews/surveys, to build on existing relevant literature such as Hou and Bonikowska’s \textit{Educational and Labour Market Outcomes of Childhood Immigrants by Admission Class}.\footnote{Hou, Feng, and Aneta Bonikowska. \textit{Educational and Labour Market Outcomes of Childhood Immigrants by Admission Class}. Statistics Canada, 25.}

\vspace{10pt}

This community-oriented research seeks to produce publishable conclusions that enhance the cultural sensitivity of migration research and advocates for socioeconomic justice of the Filipino diaspora in Philippine and Canadian policy discourse.\footnote{Especially in the context of the upcoming national elections in Canada and the Philippines.}

\vspace{10pt}

Despite challenges that affected my CGPA–labour disputes, COVID, depression, my then-subpar time management regimen–I endeavour to see this crucial research through. My academic resilience, demonstrated during the first and last few undergraduate semesters, alongside my quantitative and community-based socioeconomic research background \footnote{For details, see courses listed under “AP” on my York transcript, the “Socioeconomic Work Samples” section of my CV, my website (\url{francis-calingo.owlstown.net/}), and my GitHub portfolio (\url{https://github.com/Francis-Calingo}).}, will be conducive to this work’s success and infuse more social justice-oriented quantitative research into this program.














\end{document}
